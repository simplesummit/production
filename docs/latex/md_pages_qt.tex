Leconte runs a user interface that allows visitors to play different simulations and see how they affect the C\+PU and G\+PU on the circuit board. Leconte is written in C++ and uses Qt to create the visuals of the application.

\subsection*{What is Qt?}

\href{https://www.qt.io/}{\texttt{ Qt}} is a G\+UI S\+DK (Software Development Kit) that serves as the base for Leconte\textquotesingle{}s interface. It is used by many professional companies such as Mercedes-\/\+Benz, LG, A\+MD, and Ubuntu to create powerful and efficent applications across a range of devices.

\subsection*{Organization}

Qt is organized mainly into widgets, each with a base of Q\+Widget, that serve purposes as various from reading gamepad input to animating complex. Widgets are organized into hierarchies, with a final parent of Q\+Application, which runs in the main loop of C++.

\subsubsection*{Signals and Slots}

One of the most important things in Qt is the signal-\/and-\/slot system which enables communication between Widgets. When something happens in a widget, it can emit a {\bfseries{signal}} that is detected by a {\bfseries{slot}} in another widget. This allows for responsive behavior throughout the application. You will see this referenced throughout the document with this code\+:


\begin{DoxyCode}{0}
\DoxyCodeLine{connect(\&widget1, \&Widget1::signal, \&widget2, \&Widget2::slot);}
\end{DoxyCode}


This line of code connects the signal of {\ttfamily Widget1} to the slot of {\ttfamily Widget2}. This technique is used to great effect throughout the application.

\subsection*{Leconte\textquotesingle{}s Organization}

Leconte is split into a couple different elements. There are five Pages, widgets which hold the contents of what is on the screen. (Including Simulations) There is also the \mbox{\hyperlink{classStack}{Stack}} class, where the Pages are combined to create a dynamic navigation environment.

\subsubsection*{Pages}

\paragraph*{\mbox{\hyperlink{classHomePage}{Home\+Page}}}

The \mbox{\hyperlink{classHomePage}{Home\+Page}} class holds, well, the home screen for the application. It is the most complex part of the application, creating four diffferent interactive cards and executing animations for each one of them. It has some specific Gamepad controls and many slots that are used in the \mbox{\hyperlink{classStack}{Stack}} class.

\mbox{\hyperlink{classHomePage}{Home\+Page}} also holds the navigation for the application, which is described in more detail here.

\paragraph*{\mbox{\hyperlink{classSplashScreen}{Splash\+Screen}}}

\mbox{\hyperlink{classSplashScreen}{Splash\+Screen}} is a wrapper of the Q\+Splash\+Screen widget. It adds a custom fade-\/out animation, a background image, and text to show a message.

This class is the first thing that appears on the monitor when you boot up the application. {\ttfamily \mbox{\hyperlink{main_8cpp}{main.\+cpp}}} uses a \href{https://doc.qt.io/qt-5/qtimer.html\#singleShot}{\texttt{ {\ttfamily single-\/shot Q\+Timer}}} that lasts for 4.\+5 seconds, which matches up with the animation curve of \mbox{\hyperlink{classSplashScreen}{Splash\+Screen}}, seen above.

\paragraph*{\mbox{\hyperlink{classPaint}{Paint}}, \mbox{\hyperlink{classSmoke}{Smoke}}, and \mbox{\hyperlink{classFluid}{Fluid}}}

The \mbox{\hyperlink{classPaint}{Paint}}, \mbox{\hyperlink{classSmoke}{Smoke}}, and \mbox{\hyperlink{classFluid}{Fluid}} classes are all simulations. All inherit from the \mbox{\hyperlink{classSimulation}{Simulation}} abstract class, which holds the basic building blocks of all the pages. For example, it holds the {\ttfamily \mbox{\hyperlink{classSimulation_a36aefdee44fabe9b8070363ca9eb80a7}{Simulation\+::update\+\_\+cpu()}}} method, which returns the current C\+PU temperature.

Each of the simulations have a graph which displays the C\+PU and G\+PU temperature for the system, as to show any change created by the applications. Each graph is an object of the \mbox{\hyperlink{classGraph}{Graph}} class, which inherits from \href{https://www.qcustomplot.com/}{\texttt{ Q\+Custom\+Plot}}. For more info on how Leconte reads the temperature, see here.

\subsubsection*{For more info on the \mbox{\hyperlink{classStack}{Stack}} class, see it\textquotesingle{}s dedicated page.}