\href{http://www.youtube.com/watch?v=p5U4NgVGAwg}{\texttt{ }}

\subsubsection*{Background}

Neural\+Paint is a G\+UI created to interface with a machine learning program called S\+P\+A\+DE (short for Spatially-\/\+Adaptive Normalization) created by N\+V\+I\+D\+IA. S\+P\+A\+DE is an implementation of a technique known as {\itshape semantic image synthesis} with spatially-\/adaptive normalization which takes in hand-\/drawn images as input and outputs photorealistic interpretations of those images. The paper covering this technique can be found \href{https://nvlabs.github.io/SPADE/}{\texttt{ here}}.

\subsubsection*{Install Project Dependencies}


\begin{DoxyCode}{0}
\DoxyCodeLine{sudo apt-get install -y libblas-dev liblapack-dev libatlas-base-dev gfortran libjpeg-dev zlib1g-dev libfreetype6-dev}
\end{DoxyCode}


\subsubsection*{Frontend Installation and Setup}

The Deep Painting application has both a frontend javascript G\+UI drawing application and a backend python Flask A\+PI to interface with the python S\+P\+A\+DE code.

If you have already cloned the main repository containing all threee applications ensure that you have pulled the git submodules with \textquotesingle{}\$ git submodule update --init --recursive\textquotesingle{} from within the main project directory. This will update the G\+UI, smoke, particles, and painting applications.

Next, change directories into the painting frontend directory {\ttfamily $\sim$/production/painting/frontend} and run the following commands.


\begin{DoxyCode}{0}
\DoxyCodeLine{\# Deep Painting Requires node version 8, ionic version 3, and cordova version 8.0.1}
\DoxyCodeLine{}
\DoxyCodeLine{\# Uninstall to avoid conflicts if not using NVM}
\DoxyCodeLine{npm uninstall -g cordova}
\DoxyCodeLine{npm uninstall -g ionic}
\DoxyCodeLine{}
\DoxyCodeLine{\# Install correct versions}
\DoxyCodeLine{npm install -g ionic@3.9.2}
\DoxyCodeLine{npm install -g cordova@8}
\DoxyCodeLine{}
\DoxyCodeLine{\# Install project dependencies}
\DoxyCodeLine{cd ~/production/painting/frontend}
\DoxyCodeLine{npm install}
\DoxyCodeLine{}
\DoxyCodeLine{\# Build production version to www directory}
\DoxyCodeLine{ionic cordova build browser --prod}
\end{DoxyCode}


\subsubsection*{Backend Installation and Setup}

Change directories into server folder with {\ttfamily cd $\sim$/production/painting/frontend} and complete the following directions.

\paragraph*{Install Pytorch}

If you are running this project on N\+V\+I\+D\+IA Jetson follow the below steps. Else, install Pytorch based on instructions found on the \href{https://pytorch.org/get-started/locally/}{\texttt{ Py\+Torch Website}}


\begin{DoxyCode}{0}
\DoxyCodeLine{\# Choose an install location }
\DoxyCodeLine{mkdir [install location]}
\DoxyCodeLine{cd [install location]i}
\DoxyCodeLine{}
\DoxyCodeLine{git clone --recursive ttp://github.com/pytorch/pytorch}
\DoxyCodeLine{cd pytorch}
\DoxyCodeLine{}
\DoxyCodeLine{\# These are to avoid Pytorch CUDA errors when running CUDA 10. }
\DoxyCodeLine{export USE\_NCCL=0}
\DoxyCodeLine{export USE\_DISTRIBUTED=0}
\DoxyCodeLine{export TORCH\_CUDA\_ARCH\_LIST="5.3;6.2;7.2"i}
\DoxyCodeLine{}
\DoxyCodeLine{sudo apt-get install python3-pip cmake}
\DoxyCodeLine{sudo python3 -m pip install -U pip}
\DoxyCodeLine{sudo python3 -m install -U setuptools}
\DoxyCodeLine{sudo python3 -m install -r requirements.txt}
\DoxyCodeLine{sudo python3 -m pip install scikit-build --user}
\DoxyCodeLine{sudo python3 -m pip install ninja --useri}
\DoxyCodeLine{}
\DoxyCodeLine{\# This  will take a long time}
\DoxyCodeLine{python3 setup.py bdist\_wheel}
\DoxyCodeLine{sudo python3 setup.py install --user}
\end{DoxyCode}


\paragraph*{Install Pillow}

If you are running on an N\+V\+I\+D\+IA Jetson or any other A\+RM device there may not be a pip wheel available to you for Pillow. To fix this we must build Pillow from source.


\begin{DoxyCode}{0}
\DoxyCodeLine{cd [install location]}
\DoxyCodeLine{git clone https://github.com/python-pillow/Pillow.git}
\DoxyCodeLine{cd Pillow}
\DoxyCodeLine{sudo python3 setup.py install}
\end{DoxyCode}


\paragraph*{Install remaining dependencies with pip}


\begin{DoxyCode}{0}
\DoxyCodeLine{cd ~/production/painting/server}
\DoxyCodeLine{sudo python3 -m pip install -r requirements.txt}
\end{DoxyCode}


\subsubsection*{Run Frontend and Backend}

\paragraph*{Running frontend}


\begin{DoxyCode}{0}
\DoxyCodeLine{cd ~/production/painting/frontend/www}
\DoxyCodeLine{python3 -m http.server 8000}
\end{DoxyCode}
 \paragraph*{Running Backend}


\begin{DoxyCode}{0}
\DoxyCodeLine{cd ~/production/painting/server}
\DoxyCodeLine{sudo python3 server.py}
\end{DoxyCode}


\subsubsection*{Running Frontend and Backend as Background Services}

Both the backend and frontend of the Deep Painting application can be run persistently with minimal resource usage as lost as the server does not have images in its que. This can best be accomplished through the use of supervisord due to its simplicity and dependability when it comes to python programs.

\paragraph*{Three System Services Must Be Created}


\begin{DoxyItemize}
\item Supervisord service
\begin{DoxyItemize}
\item Webpage G\+UI (frontend) service
\item S\+P\+A\+DE (backend) service
\end{DoxyItemize}
\end{DoxyItemize}

\paragraph*{Supervisord Installation}


\begin{DoxyCode}{0}
\DoxyCodeLine{sudo apt-get install supervisor}
\DoxyCodeLine{sudo supervisord}
\end{DoxyCode}
 \paragraph*{Ensure that Supervisord is running}


\begin{DoxyCode}{0}
\DoxyCodeLine{\$ sudo service supervisor status}
\end{DoxyCode}
 \paragraph*{Create Frontend and Backend Services}


\begin{DoxyCode}{0}
\DoxyCodeLine{sudo vi /etc/supervisor/conf.d/frontend.conf}
\end{DoxyCode}
 Paste the following 
\begin{DoxyCode}{0}
\DoxyCodeLine{[program:frontend]}
\DoxyCodeLine{command/usr/bin/sudo /usr/bin/python3 -m http.server 8000}
\DoxyCodeLine{directory=/home/[Username]/production/painting/frontend}
\DoxyCodeLine{stdout\_logfile=/home/[Username]/production/logs/frontend.txt}
\DoxyCodeLine{redirect\_stderr=true}
\DoxyCodeLine{autostart=true}
\DoxyCodeLine{autorestart=true}
\end{DoxyCode}
 
\begin{DoxyCode}{0}
\DoxyCodeLine{sudo vi /etc/supervisor/conf.d/spade.conf}
\end{DoxyCode}
 Paste the following 
\begin{DoxyCode}{0}
\DoxyCodeLine{[program:spade]}
\DoxyCodeLine{user=node-5}
\DoxyCodeLine{directory=/home/node-5/production/painting/server}
\DoxyCodeLine{command=sudo python3 server.py}
\DoxyCodeLine{autostart=true}
\DoxyCodeLine{autorestart=true}
\end{DoxyCode}
 Now that you have created both services you need to activiate them using supervisor 
\begin{DoxyCode}{0}
\DoxyCodeLine{sudo supervisorctl}
\DoxyCodeLine{supervisor > reread}
\DoxyCodeLine{supervisor > add test\_process}
\DoxyCodeLine{supervisor > status}
\end{DoxyCode}


\paragraph*{Supervisord Web Interface}

If you want you can monitor your services in using Supervisord\textquotesingle{}s web interface.

To enable this add the following config to the config file found in {\ttfamily /etc/supervisor/supervisord.conf}


\begin{DoxyCode}{0}
\DoxyCodeLine{[inet\_http\_server]}
\DoxyCodeLine{port=*:9001}
\DoxyCodeLine{username=testuser}
\DoxyCodeLine{password=testpass}
\end{DoxyCode}
 