A user interacts with Leconte through a Logitech Gamepad F310 controller. Detecting and using the unique inputs of a controller was a challenge, but one that we overcame.

\subsubsection*{Q\+Gamepad}

Qt has its own library for reading gamepad input called \href{https://doc.qt.io/qt-5/qgamepad.html}{\texttt{ Q\+Gamepad}}. Leconte uses this to parse input and navigate.

Q\+Gamepad is mostly used in {\ttfamily \mbox{\hyperlink{stack_8cpp}{stack.\+cpp}}}\+: 
\begin{DoxyCode}{0}
\DoxyCodeLine{\textcolor{preprocessor}{\#include <QtGamepad/QGamepad>}}
\DoxyCodeLine{...}
\DoxyCodeLine{auto gamepads = QGamepadManager::instance()->connectedGamepads();}
\DoxyCodeLine{\textcolor{keywordflow}{if} (gamepads.isEmpty()) \{}
\DoxyCodeLine{    \textcolor{keywordflow}{return};}
\DoxyCodeLine{\}}
\DoxyCodeLine{m\_gamepad = \textcolor{keyword}{new} QGamepad(*gamepads.begin(), \textcolor{keyword}{this});  }
\end{DoxyCode}


This code instantiates the gamepad, using Q\+Gamepad\+Manager to see if there are any gamepads plugged in. To share input between widgets, \mbox{\hyperlink{classStack}{Stack}} uses the signal-\/and-\/slot system from Qt.

Here are two examples\+: 
\begin{DoxyCode}{0}
\DoxyCodeLine{connect(m\_gamepad, \&QGamepad::buttonGuideChanged, \textcolor{keyword}{this}, \&\mbox{\hyperlink{classStack_a2f8a3dee32407abe994b7332282d03de}{Stack::closeApp}});}
\DoxyCodeLine{connect(m\_gamepad, \&QGamepad::axisLeftXChanged, homePage, \&\mbox{\hyperlink{classHomePage_a63840fcd32af69c7d788ac93f6def734}{HomePage::getLeftX}});}
\end{DoxyCode}


The first example shows a button change. When the Guide button (the home button) is pressed or released, a boolean is emitted, indicating the state the button has. \mbox{\hyperlink{classStack}{Stack}} has a slot to parse this an end the application.


\begin{DoxyCode}{0}
\DoxyCodeLine{}
\DoxyCodeLine{\textcolor{keywordtype}{void} \mbox{\hyperlink{classStack_a2f8a3dee32407abe994b7332282d03de}{Stack::closeApp}}(\textcolor{keywordtype}{bool} \_isChanged) \{}
\DoxyCodeLine{    \textcolor{keywordflow}{if}(\_isChanged) \{}
\DoxyCodeLine{        QApplication::exit();}
\DoxyCodeLine{    \}}
\DoxyCodeLine{\}}
\end{DoxyCode}


The second example shows an axis change, which emits a double value with a range of -\/1 to 1. This is used by \mbox{\hyperlink{classHomePage}{Home\+Page}} to navigate through the application. To see how it does this, and for a look at the slot, see the Navigation page.

\subsubsection*{Future}

As the application becomes more complex, we will add signals and slots for the gamepad to enable more functionality. 